\PassOptionsToPackage{unicode=true}{hyperref} % options for packages loaded elsewhere
\PassOptionsToPackage{hyphens}{url}
%
\documentclass[12pt,ignorenonframetext,aspectratio=169]{beamer}
\usepackage{pgfpages}
\setbeamertemplate{caption}[numbered]
\setbeamertemplate{caption label separator}{: }
\setbeamercolor{caption name}{fg=normal text.fg}
\beamertemplatenavigationsymbolsempty
% Prevent slide breaks in the middle of a paragraph:
\widowpenalties 1 10000
\raggedbottom
\setbeamertemplate{part page}{
\centering
\begin{beamercolorbox}[sep=16pt,center]{part title}
  \usebeamerfont{part title}\insertpart\par
\end{beamercolorbox}
}
\setbeamertemplate{section page}{
\centering
\begin{beamercolorbox}[sep=12pt,center]{part title}
  \usebeamerfont{section title}\insertsection\par
\end{beamercolorbox}
}
\setbeamertemplate{subsection page}{
\centering
\begin{beamercolorbox}[sep=8pt,center]{part title}
  \usebeamerfont{subsection title}\insertsubsection\par
\end{beamercolorbox}
}
\AtBeginPart{
  \frame{\partpage}
}
\AtBeginSection{
  \ifbibliography
  \else
    \frame{\sectionpage}
  \fi
}
\AtBeginSubsection{
  \frame{\subsectionpage}
}
\usepackage{lmodern}
\usepackage{amssymb,amsmath}
\usepackage{ifxetex,ifluatex}
\usepackage{fixltx2e} % provides \textsubscript
\ifnum 0\ifxetex 1\fi\ifluatex 1\fi=0 % if pdftex
  \usepackage[T1]{fontenc}
  \usepackage[utf8]{inputenc}
  \usepackage{textcomp} % provides euro and other symbols
\else % if luatex or xelatex
  \usepackage{unicode-math}
  \defaultfontfeatures{Ligatures=TeX,Scale=MatchLowercase}
\fi
\usetheme[]{metropolis}
% use upquote if available, for straight quotes in verbatim environments
\IfFileExists{upquote.sty}{\usepackage{upquote}}{}
% use microtype if available
\IfFileExists{microtype.sty}{%
\usepackage[]{microtype}
\UseMicrotypeSet[protrusion]{basicmath} % disable protrusion for tt fonts
}{}
\IfFileExists{parskip.sty}{%
\usepackage{parskip}
}{% else
\setlength{\parindent}{0pt}
\setlength{\parskip}{6pt plus 2pt minus 1pt}
}
\usepackage{hyperref}
\hypersetup{
            pdftitle={Computational Bayesian data analysis},
            pdfauthor={Bruno Nicenboim / Shravan Vasishth},
            pdfborder={0 0 0},
            breaklinks=true}
\urlstyle{same}  % don't use monospace font for urls
\newif\ifbibliography
\usepackage{color}
\usepackage{fancyvrb}
\newcommand{\VerbBar}{|}
\newcommand{\VERB}{\Verb[commandchars=\\\{\}]}
\DefineVerbatimEnvironment{Highlighting}{Verbatim}{commandchars=\\\{\}}
% Add ',fontsize=\small' for more characters per line
\usepackage{framed}
\definecolor{shadecolor}{RGB}{248,248,248}
\newenvironment{Shaded}{\begin{snugshade}}{\end{snugshade}}
\newcommand{\AlertTok}[1]{\textcolor[rgb]{0.94,0.16,0.16}{#1}}
\newcommand{\AnnotationTok}[1]{\textcolor[rgb]{0.56,0.35,0.01}{\textbf{\textit{#1}}}}
\newcommand{\AttributeTok}[1]{\textcolor[rgb]{0.77,0.63,0.00}{#1}}
\newcommand{\BaseNTok}[1]{\textcolor[rgb]{0.00,0.00,0.81}{#1}}
\newcommand{\BuiltInTok}[1]{#1}
\newcommand{\CharTok}[1]{\textcolor[rgb]{0.31,0.60,0.02}{#1}}
\newcommand{\CommentTok}[1]{\textcolor[rgb]{0.56,0.35,0.01}{\textit{#1}}}
\newcommand{\CommentVarTok}[1]{\textcolor[rgb]{0.56,0.35,0.01}{\textbf{\textit{#1}}}}
\newcommand{\ConstantTok}[1]{\textcolor[rgb]{0.00,0.00,0.00}{#1}}
\newcommand{\ControlFlowTok}[1]{\textcolor[rgb]{0.13,0.29,0.53}{\textbf{#1}}}
\newcommand{\DataTypeTok}[1]{\textcolor[rgb]{0.13,0.29,0.53}{#1}}
\newcommand{\DecValTok}[1]{\textcolor[rgb]{0.00,0.00,0.81}{#1}}
\newcommand{\DocumentationTok}[1]{\textcolor[rgb]{0.56,0.35,0.01}{\textbf{\textit{#1}}}}
\newcommand{\ErrorTok}[1]{\textcolor[rgb]{0.64,0.00,0.00}{\textbf{#1}}}
\newcommand{\ExtensionTok}[1]{#1}
\newcommand{\FloatTok}[1]{\textcolor[rgb]{0.00,0.00,0.81}{#1}}
\newcommand{\FunctionTok}[1]{\textcolor[rgb]{0.00,0.00,0.00}{#1}}
\newcommand{\ImportTok}[1]{#1}
\newcommand{\InformationTok}[1]{\textcolor[rgb]{0.56,0.35,0.01}{\textbf{\textit{#1}}}}
\newcommand{\KeywordTok}[1]{\textcolor[rgb]{0.13,0.29,0.53}{\textbf{#1}}}
\newcommand{\NormalTok}[1]{#1}
\newcommand{\OperatorTok}[1]{\textcolor[rgb]{0.81,0.36,0.00}{\textbf{#1}}}
\newcommand{\OtherTok}[1]{\textcolor[rgb]{0.56,0.35,0.01}{#1}}
\newcommand{\PreprocessorTok}[1]{\textcolor[rgb]{0.56,0.35,0.01}{\textit{#1}}}
\newcommand{\RegionMarkerTok}[1]{#1}
\newcommand{\SpecialCharTok}[1]{\textcolor[rgb]{0.00,0.00,0.00}{#1}}
\newcommand{\SpecialStringTok}[1]{\textcolor[rgb]{0.31,0.60,0.02}{#1}}
\newcommand{\StringTok}[1]{\textcolor[rgb]{0.31,0.60,0.02}{#1}}
\newcommand{\VariableTok}[1]{\textcolor[rgb]{0.00,0.00,0.00}{#1}}
\newcommand{\VerbatimStringTok}[1]{\textcolor[rgb]{0.31,0.60,0.02}{#1}}
\newcommand{\WarningTok}[1]{\textcolor[rgb]{0.56,0.35,0.01}{\textbf{\textit{#1}}}}
\usepackage{longtable,booktabs}
\usepackage{caption}
% These lines are needed to make table captions work with longtable:
\makeatletter
\def\fnum@table{\tablename~\thetable}
\makeatother
\usepackage{graphicx,grffile}
\makeatletter
\def\maxwidth{\ifdim\Gin@nat@width>\linewidth\linewidth\else\Gin@nat@width\fi}
\def\maxheight{\ifdim\Gin@nat@height>\textheight\textheight\else\Gin@nat@height\fi}
\makeatother
% Scale images if necessary, so that they will not overflow the page
% margins by default, and it is still possible to overwrite the defaults
% using explicit options in \includegraphics[width, height, ...]{}
\setkeys{Gin}{width=\maxwidth,height=\maxheight,keepaspectratio}
\setlength{\emergencystretch}{3em}  % prevent overfull lines
\providecommand{\tightlist}{%
  \setlength{\itemsep}{0pt}\setlength{\parskip}{0pt}}
\setcounter{secnumdepth}{5}

% set default figure placement to htbp
\makeatletter
\def\fps@figure{htbp}
\makeatother

  \setbeamercolor{frametitle}{bg=gray}
  \hypersetup{colorlinks,citecolor=orange,filecolor=red,linkcolor=brown,urlcolor=blue}
% \setsansfont[BoldFont={FiraSans-Bold.ttf}]{FiraSans-Light.ttf}
% \setmonofont{FiraMono-Regular.ttf}
\usepackage[sfdefault]{FiraSans}
\newcommand{\hideFromPandoc}[1]{#1}
         \hideFromPandoc{
             \let\Begin\begin
             \let\End\end
         }

\setbeamerfont{caption}{size=\scriptsize}

\title{Computational Bayesian data analysis}
\providecommand{\subtitle}[1]{}
\subtitle{Stan language}
\author{Bruno Nicenboim / Shravan Vasishth}
\date{2020-03-11}

\begin{document}
\frame{\titlepage}

\begin{frame}
\tableofcontents[hideallsubsections]
\end{frame}
\begin{frame}

Stan is (mostly) written in C++ and can be accessed through several interfaces:

\begin{itemize}
\tightlist
\item
  R
\item
  Python
\item
  Matlab
\item
  Stata
\item
  bash
\item
  etc
\end{itemize}

\end{frame}

\begin{frame}[fragile]{A Stan program}
\protect\hypertarget{a-stan-program}{}

\begin{itemize}
\tightlist
\item
  usually saved as a \texttt{.stan} file
\item
  accessed through R (or other interfaces)
\item
  organized into a sequence of optional and obligatory blocks, which must be written in order.
\end{itemize}

\end{frame}

\begin{frame}[fragile]{A Stan program}
\protect\hypertarget{a-stan-program-1}{}

\tiny

\begin{verbatim}
functions {
// This is an optional block used for functions that can be used in other blocks.
}
data {
// Obligatory block that specifies the required data for the model.
}
transformed data {
// Optional block if we want to manipulate the data.
}
parameters {
// Obligatory block that specifies the model's parameters.
}
transformed parameters {
// Optional block if we want to manipulate the parameters (re-parametrize the model).
}
model {
// Obligatory block that specifies the model's likelihood and priors.
}
generated quantities {
// Optional block if we want to manipulate the output of our model.
}
\end{verbatim}

\end{frame}

\begin{frame}[fragile]{A Stan program}
\protect\hypertarget{a-stan-program-2}{}

\begin{itemize}
\tightlist
\item
  every variable used needs to be declared first with its type (real, integer, vector, matrix, etc.).
\item
  there must be a semi-colon (\texttt{;}) at the end of each line.
\end{itemize}

\end{frame}

\begin{frame}[fragile]

\begin{block}{Some examples of data type below (but check the \href{https://mc-stan.org/docs/2_22/reference-manual/data-types-chapter.html}{Stan reference manual} for more):}

\begin{itemize}
\tightlist
\item
  Variable \texttt{mu} contains a real number, either positive or negative:
\end{itemize}

\texttt{real\ mu;}

\begin{itemize}
\tightlist
\item
  Variable \texttt{X} contains a real number that is bounded between two numbers (or by only one). Suppose \texttt{X} is some type of measurement that can only be between 0 and 1000. (We can add lower and/or upper to any type.)
\end{itemize}

\texttt{real\textless{}lower\ =\ 0,\ upper\ =\ 1000\textgreater{}\ \ X;}

\end{block}

\end{frame}

\begin{frame}[fragile]

\begin{itemize}
\tightlist
\item
  Variable \texttt{N} contains integers, such as the number of observations (which should be \(>0\)):
\end{itemize}

\texttt{int\textless{}lower\ =\ 0\textgreater{}\ \ N;}

\begin{itemize}
\tightlist
\item
  For vectors, row\_vectors and matrices, we must define the number of elements (in each margin) that they will contain. This number can be defined earlier (such as \texttt{N}, the number of observations, in our example). (In Stan, the values inside a vectors and matrices are always \emph{real}.)
\end{itemize}

\texttt{vector\textless{}lower\ =\ 0\textgreater{}\ {[}N{]}\ Y;}

\texttt{row\_vector\textless{}lower\ =\ 0\textgreater{}\ {[}10{]}\ Y;}

\texttt{matrix\textless{}upper\ =\ 0\textgreater{}\ {[}3,\ J{]}\ Rho;}

\end{frame}

\begin{frame}[fragile]

\begin{itemize}
\tightlist
\item
  Any type can be converted into an array of as many dimensions as we want, even vectors and matrices. It's worthwhile to take a look at \href{https://mc-stan.org/docs/2_22/reference-manual/array-data-types-section.html}{Array Data Types section of the Stan reference manual}.
\end{itemize}

\texttt{real\ mu{[}2{]};\ \ //\ one\ dimension,\ two\ places}

\texttt{int\textless{}lower\ =\ 0\textgreater{}\ \ N{[}x,y,z{]};\ //\ 3\ dimensions,\ with\ x,\ y,\ z\ places.}

\texttt{vector\textless{}lower\ =\ 0\textgreater{}\ {[}N{]}\ Y{[}2{]};\ //\ array\ of\ one\ dimensions\ that\ contain\ 2\ vectors\ of\ N\ places}

\end{frame}

\begin{frame}{Example: Cloze probability with Stan (Binomial likelihood)}
\protect\hypertarget{sec:clozestan}{}

\begin{block}{We want to derive the posterior distribution of the Cloze probability of \emph{``umbrella''}, \(\theta\):}

\begin{itemize}
\tightlist
\item
  Data: a word (e.g., \emph{``umbrella''}) was answered 80 out of 100 times,
\item
  Likelihood: a binomial distribution
\item
  Prior for \(\theta\): \(Beta(a=4,b=4)\)
\end{itemize}

\end{block}

\end{frame}

\begin{frame}[fragile]{Example: Cloze probability with Stan (Binomial likelihood)}
\protect\hypertarget{sec:clozestan}{}

\scriptsize

\begin{verbatim}
data {
  int<lower = 1> N;  // Total number of answers 
  int<lower = 0, upper = N> k;  // Number of times "umbrella" was answered
}
parameters {
  // theta is a probability, it has to be constrained between 0 and 1
  real<lower = 0, upper = 1> theta;
}
model {
  // Prior on theta:
  target += beta_lpdf(theta | 4, 4); 
  // Likelihood:
  target += binomial_lpmf(k | N, theta); 
}
\end{verbatim}

\normalsize

(*) every statement with \texttt{target\ +=} increments the unnormalized \emph{log} posterior probability.

\end{frame}

\begin{frame}[fragile]

\begin{itemize}
\tightlist
\item
  Save the previous model as \texttt{stan\_models/binomial.stan} (don't run it in R).
\item
  Use the following code to call the model from R:
\end{itemize}

\begin{Shaded}
\begin{Highlighting}[]
\KeywordTok{library}\NormalTok{(rstan)}
\KeywordTok{options}\NormalTok{(}\DataTypeTok{mc.cores =}\NormalTok{ parallel}\OperatorTok{::}\KeywordTok{detectCores}\NormalTok{())}
\NormalTok{lst_cloze_data <-}\StringTok{ }\KeywordTok{list}\NormalTok{(}\DataTypeTok{k =} \DecValTok{80}\NormalTok{, }\DataTypeTok{N =} \DecValTok{100}\NormalTok{)}
\CommentTok{# Fit the model with the default values of number}
\CommentTok{# of chains and iterations (chains = 4,iter = 2000)}
\NormalTok{fit_cloze <-}\StringTok{ }\KeywordTok{stan}\NormalTok{(}
  \DataTypeTok{file =} \StringTok{"stan_models/binomial_cloze.stan"}\NormalTok{,}
  \DataTypeTok{data =}\NormalTok{ lst_cloze_data}
\NormalTok{)}
\end{Highlighting}
\end{Shaded}

\end{frame}

\begin{frame}[fragile]

\scriptsize

\begin{Shaded}
\begin{Highlighting}[]
\NormalTok{fit_cloze}
\end{Highlighting}
\end{Shaded}

\begin{verbatim}
## Inference for Stan model: binomial_cloze.
## 4 chains, each with iter=2000; warmup=1000; thin=1; 
## post-warmup draws per chain=1000, total post-warmup draws=4000.
## 
##        mean se_mean   sd  2.5%   25%   50%  75%   98%
## theta  0.78    0.00 0.04  0.69  0.75  0.78  0.8  0.85
## lp__  -5.05    0.02 0.73 -7.15 -5.23 -4.77 -4.6 -4.54
##       n_eff Rhat
## theta  1412    1
## lp__   1922    1
## 
## Samples were drawn using NUTS(diag_e) at Wed Mar 11 15:18:52 2020.
## For each parameter, n_eff is a crude measure of effective sample size,
## and Rhat is the potential scale reduction factor on split chains (at 
## convergence, Rhat=1).
\end{verbatim}

\normalsize

\end{frame}

\begin{frame}[fragile]

\texttt{bayesplot} (Gabry and Mahr 2019) is a wrapper around \texttt{ggplot2} (Wickham et al. 2019) and has several convenient functions to plot the samples (see their \href{https://mc-stan.org/bayesplot/articles/plotting-mcmc-draws.html}{vignette}).

\scriptsize

\begin{Shaded}
\begin{Highlighting}[]
\KeywordTok{library}\NormalTok{(bayesplot)}
\CommentTok{# We need to convert the fit to plot it with bayesplot functions}
\NormalTok{df_fit_cloze <-}\StringTok{ }\KeywordTok{as.data.frame}\NormalTok{(fit_cloze)}
\CommentTok{# Bayes plot functions start with mcmc_}
\KeywordTok{mcmc_dens}\NormalTok{(df_fit_cloze, }\DataTypeTok{pars =} \StringTok{"theta"}\NormalTok{) }\OperatorTok{+}
\StringTok{  }\KeywordTok{geom_vline}\NormalTok{(}\DataTypeTok{xintercept =} \KeywordTok{mean}\NormalTok{(df_fit_cloze}\OperatorTok{$}\NormalTok{theta))}
\end{Highlighting}
\end{Shaded}

\includegraphics{03b-rstan_files/figure-beamer/unnamed-chunk-3-1.pdf}

\normalsize

\end{frame}

\begin{frame}{A more complex Stan model}
\protect\hypertarget{a-more-complex-stan-model}{}

\begin{equation}
\begin{aligned}
rt_n &\sim LogNormal(\mu, \sigma)\\
\mu &\sim Normal(6, 1.5) \\
\sigma &\sim Normal_+(0, 1) 
\end{aligned}
\label{eq:infrtpriors}
\end{equation}

\end{frame}

\begin{frame}[fragile]{A more complex Stan model}
\protect\hypertarget{a-more-complex-stan-model-1}{}

\scriptsize

\begin{verbatim}
data {
  int<lower=1> N_obs;
  vector[N_obs] rt;
}
parameters {
  real<lower=0> sigma;
  real mu;
}
model {
  target += normal_lpdf(mu | 6, 1.5);
  target += normal_lpdf(sigma | 0, 1)  -
    normal_lccdf(0 | 0, 1);
  target += lognormal_lpdf(rt | mu, sigma); 
}
generated quantities {
  vector[N_obs] rt_sim;
  // posterior predictive
  for(i in 1:N_obs) {
    rt_sim[i] = lognormal_rng(mu, sigma);
  }
}
\end{verbatim}

\normalsize

\end{frame}

\begin{frame}[fragile]

\scriptsize

\begin{Shaded}
\begin{Highlighting}[]
\KeywordTok{library}\NormalTok{(dplyr)}
\KeywordTok{library}\NormalTok{(readr)}
\NormalTok{df_noreading_data <-}
\StringTok{  }\KeywordTok{read_csv}\NormalTok{(}\StringTok{"./data/button_press.csv"}\NormalTok{)}
\NormalTok{lst_noreading <-}\StringTok{ }\KeywordTok{list}\NormalTok{(}
  \DataTypeTok{N_obs =} \KeywordTok{nrow}\NormalTok{(df_noreading_data),}
  \DataTypeTok{rt =}\NormalTok{ df_noreading_data}\OperatorTok{$}\NormalTok{rt}
\NormalTok{)}

\NormalTok{fit_lognormal_reading <-}\StringTok{ }\KeywordTok{stan}\NormalTok{(}\StringTok{"stan_models/lognormal.stan"}\NormalTok{,}
  \DataTypeTok{data =}\NormalTok{ lst_noreading}
\NormalTok{)}
\end{Highlighting}
\end{Shaded}

\normalsize

\end{frame}

\begin{frame}[fragile]

\scriptsize

\begin{Shaded}
\begin{Highlighting}[]
\KeywordTok{print}\NormalTok{(fit_lognormal_reading, }\DataTypeTok{pars =} \KeywordTok{c}\NormalTok{(}\StringTok{"sigma"}\NormalTok{, }\StringTok{"mu"}\NormalTok{))}
\end{Highlighting}
\end{Shaded}

\begin{verbatim}
## Inference for Stan model: lognormal.
## 4 chains, each with iter=2000; warmup=1000; thin=1; 
## post-warmup draws per chain=1000, total post-warmup draws=4000.
## 
##       mean se_mean   sd 2.5%  25%  50%  75%  98% n_eff
## sigma 0.13       0 0.01 0.13 0.13 0.13 0.14 0.15  3081
## mu    5.12       0 0.01 5.10 5.11 5.12 5.12 5.13  4482
##       Rhat
## sigma    1
## mu       1
## 
## Samples were drawn using NUTS(diag_e) at Wed Mar 11 15:46:49 2020.
## For each parameter, n_eff is a crude measure of effective sample size,
## and Rhat is the potential scale reduction factor on split chains (at 
## convergence, Rhat=1).
\end{verbatim}

\normalsize

\end{frame}

\begin{frame}[fragile]{Extracting Stan code from brms}
\protect\hypertarget{extracting-stan-code-from-brms}{}

\scriptsize

\begin{Shaded}
\begin{Highlighting}[]
\KeywordTok{library}\NormalTok{(brms)}
\NormalTok{fit_press_ln <-}\StringTok{ }\KeywordTok{brm}\NormalTok{(rt }\OperatorTok{~}\StringTok{ }\DecValTok{1}\NormalTok{,}
  \DataTypeTok{data =}\NormalTok{ df_noreading_data,}
  \DataTypeTok{family =} \KeywordTok{lognormal}\NormalTok{(),}
  \DataTypeTok{prior =} \KeywordTok{c}\NormalTok{(}
    \KeywordTok{prior}\NormalTok{(}\KeywordTok{normal}\NormalTok{(}\DecValTok{6}\NormalTok{, }\FloatTok{1.5}\NormalTok{), }\DataTypeTok{class =}\NormalTok{ Intercept),}
    \KeywordTok{prior}\NormalTok{(}\KeywordTok{normal}\NormalTok{(}\DecValTok{0}\NormalTok{, }\DecValTok{1}\NormalTok{), }\DataTypeTok{class =}\NormalTok{ sigma)}
\NormalTok{  )}
\NormalTok{)}
\end{Highlighting}
\end{Shaded}

\normalsize

\end{frame}

\begin{frame}[fragile]

\begin{block}{Extracting code when the model was run:}

\vspace{1cm}

\small

\begin{Shaded}
\begin{Highlighting}[]
\KeywordTok{stancode}\NormalTok{(fit_press_ln)}
\end{Highlighting}
\end{Shaded}

\normalsize

\end{block}

\end{frame}

\begin{frame}[fragile]

\scriptsize

\begin{verbatim}
## // generated with brms 2.12.0 
##  functions { 
##  } 
##  data { 
##    int<lower=1> N;  // number of observations 
##    vector[N] Y;  // response variable 
##    int prior_only;  // should the likelihood be ignored? 
##  } 
##  transformed data { 
##  } 
##  parameters { 
##    real Intercept;  // temporary intercept for centered predictors 
##    real<lower=0> sigma;  // residual SD 
##  } 
##  transformed parameters { 
##  } 
##  model {
\end{verbatim}

\normalsize

\textbf{\ldots{}}

\end{frame}

\begin{frame}[fragile]

\textbf{\ldots{}}

\scriptsize

\begin{verbatim}
## } 
##  model { 
##    // initialize linear predictor term 
##    vector[N] mu = Intercept + rep_vector(0, N); 
##    // priors including all constants 
##    target += normal_lpdf(Intercept | 6, 1.5); 
##    target += normal_lpdf(sigma | 0, 1) 
##      - 1 * normal_lccdf(0 | 0, 1); 
##    // likelihood including all constants 
##    if (!prior_only) { 
##      target += lognormal_lpdf(Y | mu, sigma); 
##    } 
##  } 
##  generated quantities { 
##    // actual population-level intercept 
##    real b_Intercept = Intercept; 
##  }
\end{verbatim}

\normalsize

\end{frame}

\begin{frame}[fragile]

\begin{block}{Extracting code before the model was run:}

\vspace{1cm}

\scriptsize

\begin{Shaded}
\begin{Highlighting}[]
\KeywordTok{make_stancode}\NormalTok{(rt }\OperatorTok{~}\StringTok{ }\DecValTok{1}\NormalTok{,}
  \DataTypeTok{data =}\NormalTok{ df_noreading_data,}
  \DataTypeTok{family =} \KeywordTok{lognormal}\NormalTok{(),}
  \DataTypeTok{prior =} \KeywordTok{c}\NormalTok{(}
    \KeywordTok{prior}\NormalTok{(}\KeywordTok{normal}\NormalTok{(}\DecValTok{6}\NormalTok{, }\FloatTok{1.5}\NormalTok{), }\DataTypeTok{class =}\NormalTok{ Intercept),}
    \KeywordTok{prior}\NormalTok{(}\KeywordTok{normal}\NormalTok{(}\DecValTok{0}\NormalTok{, }\DecValTok{1}\NormalTok{), }\DataTypeTok{class =}\NormalTok{ sigma)}
\NormalTok{  )}
\NormalTok{)}
\end{Highlighting}
\end{Shaded}

\normalsize

\end{block}

\end{frame}

\begin{frame}[fragile]

\begin{block}{Extracting the data when the model was run:}

\vspace{.5cm}

\scriptsize

\begin{Shaded}
\begin{Highlighting}[]
\NormalTok{ls_stan <-}\StringTok{ }\KeywordTok{standata}\NormalTok{(fit_press_ln)}
\NormalTok{ls_stan }\OperatorTok\StringTok{ }\KeywordTok{str}\NormalTok{()}
\end{Highlighting}
\end{Shaded}

\begin{verbatim}
## List of 5
##  $ N         : int 361
##  $ Y         : num [1:361(1d)] 141 138 128 132 126 134 163 149 133 110 ...
##  $ K         : int 1
##  $ X         : num [1:361, 1] 1 1 1 1 1 1 1 1 1 1 ...
##   ..- attr(*, "dimnames")=List of 2
##   .. ..$ : chr [1:361] "1" "2" "3" "4" ...
##   .. ..$ : chr "Intercept"
##   ..- attr(*, "assign")= int 0
##  $ prior_only: int 0
##  - attr(*, "class")= chr "standata"
\end{verbatim}

\normalsize

\end{block}

\end{frame}

\begin{frame}[fragile]

\begin{block}{Extracting the data before the model was run:}

\vspace{1cm}

\scriptsize

\begin{Shaded}
\begin{Highlighting}[]
\NormalTok{ls_stan <-}\StringTok{ }\KeywordTok{make_standata}\NormalTok{(rt }\OperatorTok{~}\StringTok{ }\DecValTok{1}\NormalTok{,}
  \DataTypeTok{data =}\NormalTok{ df_noreading_data,}
  \DataTypeTok{family =} \KeywordTok{lognormal}\NormalTok{(),}
  \DataTypeTok{prior =} \KeywordTok{c}\NormalTok{(}
    \KeywordTok{prior}\NormalTok{(}\KeywordTok{normal}\NormalTok{(}\DecValTok{6}\NormalTok{, }\FloatTok{1.5}\NormalTok{), }\DataTypeTok{class =}\NormalTok{ Intercept),}
    \KeywordTok{prior}\NormalTok{(}\KeywordTok{normal}\NormalTok{(}\DecValTok{0}\NormalTok{, }\DecValTok{1}\NormalTok{), }\DataTypeTok{class =}\NormalTok{ sigma)}
\NormalTok{  )}
\NormalTok{)}
\end{Highlighting}
\end{Shaded}

\normalsize

\end{block}

\end{frame}

\begin{frame}{References}
\protect\hypertarget{references}{}

\hypertarget{refs}{}
\leavevmode\hypertarget{ref-R-bayesplot}{}%
Gabry, Jonah, and Tristan Mahr. 2019. \emph{Bayesplot: Plotting for Bayesian Models}. \url{https://CRAN.R-project.org/package=bayesplot}.

\leavevmode\hypertarget{ref-R-ggplot2}{}%
Wickham, Hadley, Winston Chang, Lionel Henry, Thomas Lin Pedersen, Kohske Takahashi, Claus Wilke, Kara Woo, and Hiroaki Yutani. 2019. \emph{Ggplot2: Create Elegant Data Visualisations Using the Grammar of Graphics}. \url{https://CRAN.R-project.org/package=ggplot2}.

\end{frame}

\end{document}
